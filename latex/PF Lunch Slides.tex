\documentclass[aspectratio=43]{beamer}
% \documentclass[aspectratio=169]{beamer}

% Title --------------------------------------------
\title{Title }
\subtitle{Public Finance Lunch}
\date{October 16, 2023}
\author{Nathan Lazarus \and Ian Sapollnik }

\input{preamble.tex}

\usepackage{tikz}
\usepackage{mathdots}
\usepackage{yhmath}
\usepackage{cancel}
\usepackage{color}
\usepackage{siunitx}
\usepackage{array}
\usepackage{multirow}
\usepackage{amssymb}
\usepackage{adjustbox}
\usepackage{gensymb}
\usepackage{tabularx}
\usepackage{booktabs}
\usetikzlibrary{fadings}
\usetikzlibrary{patterns}
\usetikzlibrary{shadows.blur}
\usetikzlibrary{shapes}
\usepackage{lipsum}

% Set-up Bibliography ------------------------------
\addbibresource{references.bib}

\begin{document}

% ------------------------------------------------------------------------------
\begin{frame}
\maketitle


% \vspace{2.5mm}
% {\footnotesize $^*$ A bit of extra info here. Add an asterich to title or author}
\end{frame}
% ------------------------------------------------------------------------------

\begin{frame}{Introduction}
	\begin{itemize}
	\item Motivate that multi-regions firms' profits have to go somewhere and there are different ways of assigning them
	\item Internationally, separate accounting is the norm. Within countries, state/provincial taxes are usually apportioned
	\begin{itemize}
		\item Throw in a fun example of how Apple/Google shove IP in Ireland/Bermuda and then pay fees to 

	\end{itemize}
	\item To curtain tax avoidance, OECD has been pushing countries to implement 2 pillars that include sales factor apportionment
	\end{itemize}

\end{frame}

\begin{frame}{What is factor apportionment?}

A visual here or on the next slide would be nice!

\end{frame}

\begin{frame}{Research question: price manipulation}
\begin{itemize}
\item When taxes are apportioned 


\item Related literature here, comment on how in the past the focus has been on the quantity side (ie relocating labour, capital, etc). We would be the first to show this on the goods side
\end{itemize}


\end{frame}

\begin{frame}{Related literature}
\begin{itemize}


\item Comment on how in the past the focus has been on the quantity side (ie relocating labour, capital, etc). We would be the first to show this empirically on the goods side

\item Not a novel idea theoretically, but kind of a folk theorem: Gordon and Wilson mention this textually, but (sadly!!!!) it is a feature in the Fajgelbaum model. But a side comment and not the main issue

\item Fits in to emerging literature on tax consulting services which are currently used to game the separate accounting system. If OECD implements, these people would shift to tax advising under sales factor, and would clue into this. 


\end{itemize}

\end{frame}

\begin{frame}{Simple visual example}

This needs to be a very simple and engaging graphic which shows how a 2-state firm can maximize profits by changing its markups
\end{frame}

\begin{frame}{Our model: set up}


\end{frame}

\begin{frame}{Our model: SA}


\end{frame}

\begin{frame}{Our model: SFA}


\end{frame}

\begin{frame}{Our model: Labour}


\end{frame}

\begin{frame}{Our model: labour (monopsony)}



\end{frame}

\begin{frame}{Empirics}
\begin{itemize}
\item Talk about ideal data set/institutional setting

\item Talk about apportionment in the US and the policy changes that have occured

\item Talk about data
\end{itemize}

\end{frame}

\begin{frame}{Results}

\end{frame}


\end{document}


